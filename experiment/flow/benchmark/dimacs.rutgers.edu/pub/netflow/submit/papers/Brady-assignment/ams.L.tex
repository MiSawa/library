From htn Mon Feb 17 16:55:22 1992
Return-Path: <htn>
Received: from kato.lmsc.com by jade.rdd.lmsc.lockheed.com (4.0/SMI-4.0)
	id AA22755; Mon, 17 Feb 92 16:55:20 PST
Date: Mon, 17 Feb 92 16:55:20 PST
From: htn (H.T. Nguyen)
Message-Id: <9202180055.AA22755@jade.rdd.lmsc.lockheed.com>
To: subramon
Subject: amslatex
Status: R

%------------------------------------------------------------------------------
% Beginning of DIMACS-L.let-e
%------------------------------------------------------------------------------
Following this letter are three files (DIMACS-L.STY, DIMACS-L.TEX and
DIMACS-L.CHECKLIST) to be used when preparing a paper in AMS-LaTeX, for a
Proceedings volume in the YYY series.  To make it easier to load these
files separately on your system,  they are separated by file delimiters.

The .STY file is based on AMSBOOK.STY. Instructions for using this file are
listed at the top.

The .TEX file is a sample of the macros defined in the .STY file.  Before
preparing your paper you should TeX this file and print it to test the macros
and to get instructions for their use.

The .CHECKLIST file should be printed and used to help you avoid the most
common problems.

Once your paper has been accepted for publication in DIMACS, please send the
.TEX file to our Internet address: PUB-SUBMIT@MATH.AMS.COM.  When you send the
file please be sure to include a message that identifies both the publication
series and the editor of the volume.

Should you have any technical questions, please contact our Technical Support
Group at 800-321-4AMS or 401-455-4080.  They can also be reached through
electronic mail at Internet: tech-support@math.ams.com.

%------------------------------------------------------------------------------
% End of DIMACS-L.let-e
%------------------------------------------------------------------------------
%------------------------------------------------------------------------------
% Beginning of dimacs-l.sty
%------------------------------------------------------------------------------
%%%%%%%%%%%%%%%%%%%%%%%%%%%%%%%%%%%%%%%%%%%%%%%%%%%%%%%%%%%%%%%%%%%%%
%
% This file is designed to work with AMSLaTeX version 1.1. It uses
% amsart.sty as a base and modifies page size, top matter, section
% headings, etc. to produce DIMACS style.  A logo and copyright line
% like the ones that  will appear on the final published version are
% also added.
%
% Instructions:
%
%  1. Create a file named  dimacs-l.sty ("-L" for LaTeX)
%     containing the lines from 
%      
%             %%% dimacs-l.sty 
%
%     through 
%
%             \endinput
%
%     Install the file in the same directory as amsart.sty.
%
%  2. In a document file, use \documentstyle{dimacs-l}.
%
%%%%%%%%%%%%%%%%%%%%%%%%%%%%%%%%%%%%%%%%%%%%%%%%%%%%%%%%%%%%%%%%%%%%%%%%%
%%% dimacs-l.sty
\input amsart.sty
%
\textheight=49pc  \textwidth=29pc
\headsep=14pt
%
%      In AMS production, \parskip has no stretch component.
\parskip=0pt
%
%      \Large is not defined in amsart[10].sty in order to
%      conserve font memory. But we need it here for fourteen-point
%      titles.
\def\Large{\@setsize\Large{18\p@}\xivpt\@xivpt}
%
%      Change the baselineskip from 12pt to 13pt.
\def\@normalsize{\@setsize\normalsize{13\p@}\xpt\@xpt
  \abovedisplayskip 6\p@ \@plus 6\p@
  \belowdisplayskip \abovedisplayskip
  \abovedisplayshortskip \z@ \@plus 6\p@
  \belowdisplayshortskip 3\p@ \@plus 6\p@
  \let\@listi\@listI}
%
\normalsize
%
%      \footnotesize is changed only in order to change \@listi.
%
\def\footnotesize{\@setsize\footnotesize{10\p@}\viiipt\@viiipt%
  \abovedisplayskip 5\p@ \@plus 2.5\p@ minus2.5\p@%
  \belowdisplayskip \abovedisplayskip
  \abovedisplayshortskip \z@ \@plus 2\p@%
  \belowdisplayshortskip 4\p@ \@plus 2\p@ minus2\p@%
  \def\@listi{\topsep \z@ \parsep \z@ \itemsep \z@}}
%
%      Sectioning commands
\def\section{\@startsection{section}% counter name
{1}% sectioning level
{\z@}% indent to the left of the section title
{12\p@\@plus3\p@}% vertical space above
{5\p@}% vertical space below
{\centering\bf}}% font of title
%
\def\subsection{\@startsection{subsection}% counter name
{2}% sectioning level
{\parindent}% indent to the left of the subsection title
{6\p@\@plus2\p@}% vertical space above
{-.5em}% following text is run in, after a horizontal space of this amount.
% The minus sign is to get horizontal space instead of vertical space.
{\bf}}% The font of the subsection title
%
\def\subsubsection{\@startsection {subsubsection}% counter number
{3}% sectioning level
{\parindent}% %% indent to the left of the subsubsection title
{6\p@\@plus2\p@}%  vertical space above
{-.5em}% horizontal space after (negative sign will be canceled)
{\it}}% font of subsubsection title
%
% Levels below C-head (\subsubsection) are undefined in AMS
% production documentstyles.
%
\def\paragraph{\@latexerr{\noexpand\paragraph not defined
  in this \string\documentstyle}\@eha}
\def\subparagraph{\@latexerr{\noexpand\subparagraph not defined
  in this \string\documentstyle}\@eha}
%
%      Set \topsep to 0 in \@listI.
\def\@listI{\leftmargin\leftmargini
  \parsep\z@skip \topsep\z@skip \itemsep\z@skip}
\let\@listi\@listI
\@listi
%
%      Cites in AMS proceedings volumes are typically typeset
%      in bold rather than roman.
\def\@cite#1#2{\rom{\mediumseries[{\bf#1}\if@tempswa , #2\fi]}}
%
%      No extra space around definition-style enunciations.
\def\th@definition{\theorempreskipamount\z@
    \theorempostskipamount\theorempreskipamount
    \normalshape}
%
%      Theorem Head font is small caps for proceedings volumes,
%      instead of bold.
\def\theorem@headerfont{\defaultfont\sc}
%
%      Indent instead of no indent at the beginning of a theorem:
\let\theorem@indent\indent
%
%      Use small caps instead of italic for \proofname:
\def\pf{%
  \par\topsep6\p@\@plus6\p@
  \trivlist \itemindent\normalparindent
  \item[\hskip\labelsep\sc\proofname.]\ignorespaces}
%
%      Change the running head font to \scriptsize (7pt) instead of
%      \small (8pt).
\def\ps@headings{\let\@mkboth\@gobbletwo
  \let\partmark\@gobble\let\sectionmark\@gobble
  \let\subsectionmark\@gobble
  \let\@oddfoot\@empty\let\@evenfoot\@empty%
  \def\@evenhead{\defaultfont\scriptsize
      \rlap{\thepage}\hfil
      \expandafter\uppercasetext@\expandafter{\sh@rtauthor}\hfil}%
  \def\@oddhead{\defaultfont\scriptsize \hfil
      \expandafter\uppercasetext@\expandafter{\sh@rttitle}\hfil
      \llap{\thepage}}%
}
%
\def\ps@myheadings{\let\@mkboth\@gobbletwo
  \let\@oddfoot\@empty\let\@evenfoot\@empty
  \def\@oddhead{\null\defaultfont\scriptsize\rightmark \hfil\thepage}%
  \def\@evenhead{\defaultfont\scriptsize \thepage\hfil\leftmark\null}%
}
%
%       Some computations to make the \textheight come out
%       to an even multiple of baselineskip (after taking
%       \topskip etc. into account).
\advance\textheight by -\headheight
\advance\textheight by -\headsep
\advance\textheight by -\normalbaselineskip
\advance\textheight by \topskip
%
%       We now have the right value for \textheight except that
%       we need to round it.
\advance\textheight by.5\normalbaselineskip
\divide\textheight by\normalbaselineskip
\multiply\textheight by\normalbaselineskip
%       One last note: if \baselinestretch is changed in an
%       actual document, this \textheight will most likely be
%       off and you'll get underfull vbox messages.
%
\advance\topmargin by -.25 true in
%
%      LIST ENVIRONMENTS
%
%      Change first-level `enumerate' numbering style from arabic
%      to roman numeral.
\renewcommand{\labelenumi}{(\roman{enumi})}
%
%      Change \enumerate and \itemize to increase \partopsep.
\def\enumerate{\ifnum \@enumdepth >3 \@toodeep\else
      \advance\@enumdepth \@ne
      \edef\@enumctr{enum\romannumeral\the\@enumdepth}\list
      {\csname label\@enumctr\endcsname}{\usecounter
        {\@enumctr}\partopsep6\p@\def\makelabel##1{\hss\llap{\normalshape##1}}}\fi}
%
\def\itemize{\ifnum\@itemdepth>3 \@toodeep
  \else \advance\@itemdepth\@ne
  \edef\@itemitem{labelitem\romannumeral\the\@itemdepth}%
  \list{\csname\@itemitem\endcsname}%
  {\partopsep6\p@\def\makelabel##1{\hss\llap{\normalshape##1}}}\fi}
%
%      Change the font of `References' head.
\def\thebibliography#1{\section*{\series m\sc\refname}%
%      Now we need to reset the running heads because we don't
%      wan't them to contain the font reference \series m\sc.
%      We want them to be uppercase, which is a little tricky
%      if we want to use \refname instead of an explicit value.
  \edef\@tempa{\uppercase{\noexpand\@mkboth{\refname}{\refname}}}%
  \@tempa
  \defaultfont\small\labelsep .5em\relax
  \list{\@arabic\c@enumi.}{\settowidth\labelwidth{#1.}%
  \leftmargin\labelwidth \advance\leftmargin\labelsep
  \usecounter{enumi}}%
  \sloppy \clubpenalty4000\relax \widowpenalty\clubpenalty
  \sfcode`\.\@m}
%
%      TOPMATTER 
%
%      In \@maketitle we make the following changes:
%
%      \topskip --> 10pc - headheight - headsep
%      title --> \Large (14pt), not uppercased
%      author --> aboveskip = 24pt, normalsize instead of small
%      final space is 32pt plus14pt without subtracting baselineskip
%
\def\@maketitle{%
  \defaultfont\normalsize
  \let\@makefnmark\relax  \let\@thefnmark\relax
  \ifx\@empty\@subjclass\else
   \@footnotetext{1991 {\it Mathematics Subject
     Classification}.\enspace
        \@subjclass.}\fi
  \ifx\@empty\@keywords\else
   \@footnotetext{{\it Key words and phrases.}\enspace \@keywords.}\fi
\ifx\@empty\@thanks\else
   \@footnotetext{\@thanks}\fi
  \topskip10pc % 10 pc from top of series logo to base of first title line
  \advance\topskip-\headsep \advance\topskip-\headheight
  \vtop{\centering{\Large\bf\@title\@@par}%
   \global\dimen@i\prevdepth}%
  \prevdepth\dimen@i
  \ifx\@empty\@authors
  \else
    \baselineskip24\p@
    \vtop{\@andify{ AND }\@authors
      \centering{\expandafter\uppercasetext@\expandafter
       {\@authors}\@@par}%
         \global\dimen@i\prevdepth}\relax
    \prevdepth\dimen@i
  \fi
  \ifx\@empty\@dedicatory
  \else
    \baselineskip18\p@
  \vtop{\centering{\small\it\@dedicatory\@@par}%
      \global\dimen@i\prevdepth}\prevdepth\dimen@i
  \fi
  \ifx\@empty\@date\else
  \baselineskip24\p@
    \vtop{\centering\@date\@@par
      \global\dimen@i\prevdepth}\prevdepth\dimen@i
  \fi
  \normalsize
  \vskip32\p@\@plus14\p@
  } % end \@maketitle
%
%      Make a special page style for the first page of an article,
%      to handle \serieslogo@.
\def\ps@firstpage{\ps@plain \def\@oddhead{\serieslogo@\hss}%
  \let\@evenhead\@oddhead % in case an article starts on a left-hand page
}
%
%      The \kern-\headheight is needed to offset the \vbox to
%      \headheight used in \@outputpage to typeset the running head.
%      The depth of the \serieslogo@ will be offset by a \vss in
%      that same part of \@outputage.
\def\serieslogo@{\vbox{\kern-\headheight
  \parindent\z@ \size{6}{6.5\p@}\selectfont
        DIMACS Series in Discrete Mathematics\newline
        and Theoretical Computer Science\newline
	Volume {\bf\currentvolume}, \currentyear\endgraf}}
%
%      The only change in \maketitle is \thispagestyle{firstpage}
%      instead of \thispagestyle{plain}
\def\maketitle{\par
  \@topnum\z@ % this prevents figures from falling at the top of page 1
  \ifx\@empty\sh@rtauthor \let\sh@rtauthor\sh@rttitle\fi
  \begingroup
  \@maketitle
  \endgroup
  \@andify{ AND }\sh@rtauthor
  \thispagestyle{firstpage}%
  \c@footnote\z@
  \def\do##1{\let##1\relax}%
  \do\maketitle \do\@maketitle
  \do\title \do\@xtitle \do\@title
  \do\author \do\@xauthor \do\@authors
  \do\address \do\@xaddress
  \do\email \do\@xemail \do\curraddr \do\@xcurraddr
  \do\dedicatory \do\@dedicatory
  \do\thanks \do\@thanks
  \do\keywords \do\@keywords
  \do\subjclass \do\@subjclass
  \do\@andify
}
%
\def\copyrightyear{0000}
\def\copyrightyearmodC{00}
\def\ISSN{0000-0000}
\def\copyrightprice{\$1.00\;+\;\$.25 per page}
\def\currentvolume{00}
\def\currentyear{0000}
%
%      Doing the copyright info on the first page is a little tricky. We
%      want it to come at the bottom, after any footnotes and floating
%      inserts, but before the page number.  If we put it into \@oddfoot
%      (in \ps@plain) its height will not be subtracted from the height of
%      the text and then the page number will be lower than we want.
%      Instead we put the copyright info into \@textbottom, which is normally
%      used by \raggedbottom and \flushbottom.
%
\def\raggedbottom{\typeout{\string\raggedbottom\space
  disabled; see the \noexpand\documentstyle file for details.}}
%
%      \baselineskip and \lineskip are set to 0 in LaTeX output routine,
%      so we don't need to worry about them in \@textbottom.  After the
%      first use of \@textbottom \copyrightbox@ is void, but \@textbottom
%      is still emptied out, to save a couple tokens of memory.
%
\def\@textbottom{\box\copyrightbox@ \global\let\@textbottom\@empty}
%
\newbox\copyrightbox@
\newdimen\pagetocopyright@      \pagetocopyright@=1.5pc
%
%      We set the copyright info in a box in order to measure the
%      height, because we want to subtract it from the vsize on the
%      first page. This setbox operation has to be done here
%      rather than earlier because it freezes information like ISSN
%      number.
\setbox\copyrightbox@=\vbox{\hsize\textwidth
  \parfillskip\z@ \leftskip\z@\@plus.9\textwidth
  \size{6}{6.5\p@}\selectfont
%      The spacing between the preceding text and the copyright info is
%      done with a strut of height \pagetocopyright@. (\lineskip and
%      \baselineskip are 0 in the LaTeX output routine.) The \everypar{}
%      turns off the LaTeX warning about setting text before
%      \begin{document}.
  \everypar{}\noindent\vrule\@width\z@\@height\pagetocopyright@
  \copyright\copyrightyear\ American Mathematical Society\break
  \ISSN/\copyrightyearmodC\ \copyrightprice\endgraf
%      This kern of 0pt forces the depth of the last line (if any) to be
%      added to the height of the box.
  \kern\z@}
%
%      We subtract the height of the copyright box from the first
%      page height, by way of \@preamblecmds.
%
\expandafter\def\expandafter\@preamblecmds\expandafter{%
   \@preamblecmds\advance\vsize-\ht\copyrightbox@}
%
\endinput
%------------------------------------------------------------------------------
% End of dimacs-l.sty
%------------------------------------------------------------------------------
%------------------------------------------------------------------------------
% Beginning of dimacs-l.tex
%------------------------------------------------------------------------------
% This is a sample file for use with AMS-LaTeX. It provides an example of
% how to set up a file to be typeset with AMS-LaTeX.
%%%%%%%%%%%%%%%%%%%%%%%%%%%%%%%%%%%%%%%%%%%%%%%%%%%%%%%%%%%%%%%%%%%%
%
\documentstyle{dimacs-l}

\newtheorem{lemma}{Lemma}[section]
\newtheorem{theorem}{Theorem}[section]
\newtheorem{definition}{Definition}[section]

\theoremstyle{definition}
\newtheorem{example}{Example}[section]

\numberwithin{equation}{section}

\begin{document}

\title[MAXIMAL IDEALS IN SUBALGEBRAS OF $C(X)$]{Sample Paper for DIMACS,\\
On Maximal Ideals in Subalgebras of $C(X)$}
\author[AUTHOR ONE AND AUTHOR TWO]{Author One and Author Two}
\address{Department of Mathematics, Northeastern University, Boston,
Massachusetts 02115} % Research address for author one
\curraddr{Department of Mathematics and Statistics,
Case Western Reserve University, Cleveland, Ohio 43403}
\email{XYZ@@Math.AMS.com}
\address{Mathematical Research Section, School of Mathematical Sciences,
Australian National University, Canberra ACT 2601, Australia} %address for 
%                                                             author two

\subjclass{Primary 54C40, 14E20; Secondary 46E25, 20C20}
\date{July 2, 1991}

%  \thanks will become a 1st page footnote.
\thanks{The first author was supported in part by NSF
Grant \#000000.}
\thanks{The final version of this paper will be submitted for
publication elsewhere.}

\maketitle

\begin{abstract}
This paper is a sample prepared to illustrate for authors
the use of the \AmS-\LaTeX{} Version~1.1 package and the DIMACS documentstyle.

The file used to prepare this sample is {\bf DIMACS-L.tex}; an author
should use the coding in that file as a model.
\end{abstract}

\section{Introduction}

This sample paper illustrates the use of the AMSART style file of
\AmS-\LaTeX{} Version~1.1 with additional macros for the series {\it
DIMACS Series in Discrete Mathematics and Theoretical Computer Science}.
In this sample paper, brief instructions to authors will be interspersed
with mathematical text extracted from (purposely unidentified) published
papers.  For instructions on preparing mathematical text, the author is
referred to {\it The Joy of \TeX}, second edition, by Michael Spivak
\cite{spivak:jot} and {\it \LaTeX{}: A Document Preparation System} by
Leslie Lamport \cite{lamport:latex}.

\subsection{Top matter}
The input format and content of the top matter can be best understood
by examining the first part of the sample file {\bf DIMACS-L.tex}, up
through the \verb+\begin{document}+ instruction.

The top matter includes both elements that must be input by the author and a
few that are provided automatically.  The author names and the title that are
to appear in the running heads should be input between square brackets as an
option to the \verb+\author+ and \verb+\title+ commands, respectively. The full
names and title should be used unless they require too much space; in that
event, abbreviated forms should be substituted. In the top matter, the title is
input in caps and lowercase and will be set that way.  The author names should
be input in caps and lowercase; they will automatically be set in all caps.

For each author an address should be input.  If the current address is
different than the address where the research was carried out then both
addresses are given with the current address second and coded as indicated
in this sample file.   Following these addresses, an
address for electronic mail should be given, if one exists. Note that no
abbreviations are used in addresses, and complete addresses for each author
should be entered in the order that names appear on the title page.  Addresses
are considered part of the top matter but are set at the end of the paper,
following the references.

Subject classifications (\verb+\subjclass+) and acknowledgments
(\verb+\thanks+) are part of the top matter and will appear as footnotes at the
bottom of the first page.  Subject classifications
(\verb+\subjclass+) are required.  Use the 1991 Mathematics Subject Classification
that appears in annual indexes of {\it Mathematical Reviews\/}
beginning in 1990.  (The two-digit code from the Contents is not sufficient.)

Use \verb+\thanks+ for the footnotes that appear on the first page.  It is
generally desirable not to attach footnote numbers or symbols to titles or
author names used as headings.  If a footnote applies to only one author then
include this information in the footnote.

Papers published in proceedings of conferences are often abstracts or
preliminary versions.  In such a case, include a separate \verb+\thanks+
command stating ``The final
[detailed] version of this paper will be [has been] submitted for
publication elsewhere.''  Papers that are to be considered for review
by {\it Mathematical Reviews\/} should use instead
the following statement:
``This paper is in final form and no version of it will be submitted
for publication elsewhere.''

\subsection{Fonts}
The fonts used in this paper are from the Computer Modern family; they
should be available to all authors preparing papers with these macros.
However, the final copy may be set by the AMS using other fonts.  

\subsection{A mathematical extract}
The mathematical content of this sample paper has been extracted from
published papers, with no effort made to retain any mathematical sense.
It is intended only to illustrate the recommended manner of input.

Mathematical symbols in text should always be input in math mode as
illustrated in the following paragraph.

A function is invertible in $C(X)$ if it is never zero, and in $C^*(X)$ if
it is bounded away from zero. In an arbitrary $A(X)$, of course, there
is no such description of invertibility which is independent of the 
structure of the algebra. Thus in \S 2 we associate to each noninvertible
$f\in A(X)$ a $z$-filter $\cal Z (f)$ that is a measure of where
$f$ is ``locally'' invertible in $A(X)$. This correspondence extends to
one between maximal ideals of $A(X)$ and $z$-ultrafilters on $X$.
In \S 3 we use the filters $\cal Z (f)$ to describe the intersection of 
the free maximal ideals in any algebra $A(X)$. Finally, our main result
allows us to introduce the notion of $A(X)$-compactness of which 
compactness and realcompactness are special cases. In \S 4 we show how
the Banach-Stone theorem extends to $A(X)$-compact spaces.

\section{Theorems, lemmas, and other proclamations}

Theorems and lemmas are varieties of \verb+theorem+ environments.  In this
document, a \verb+theorem+ environment called \verb+lemma+ has been created,
which is used below. Also, there is a  proof, which is in the predefined
\verb+pf+ environment.  The lemma and proof below illustrate the use of 
the \verb+enumerate+ environment. 

\begin{lemma}
Let $f, g\in  A(X)$ and let $E$, $F$ be cozero
sets in $X$.
\begin{enumerate}
\item If $f$ is $E$-regular and $F\subseteq E$, then $f$ is $F$-regular.

\item If $f$ is $E$-regular and $F$-regular, then $f$ is $E\cup F$-%
regular.

\item If $f(x)\ge c>0$ for all $x\in E$, then $f$ is $E$-regular.

\end{enumerate}
\end{lemma}

\begin{pf}
\begin{enumerate}

\item  Obvious.

\item Let $h, k\in A(X)$ satisfy $hf|_E=1$ and $kf|_F=1$. Let
$w=h+k-fhk$. Then $fw|_{E\cup F}=1$.

\item Let $h=\max\{c,f\}$. Then $h|_E=f|_E$ and $h\ge c$. So $0<h^{-1}
\le c^{-1}$. Hence $h^{-1} \in C^*(X)\subseteq A(X)$, and 
$h^{-1} f|_E=1$. 

\end{enumerate}
\end{pf}

Another \verb+theorem+-type environment was defined at the beginning of this
document, called \verb+definition+. Here is an example of it:

\begin{definition}
For $f\in A(X)$, we define
\begin{equation}
\cal Z (f)=\{E\in Z[X]\: \text{$f$ is $E^c$-regular}\}.
\end{equation}
\end{definition}

\section{Roman type}

Numbers, punctuation, (parentheses), [brackets], $\{$braces$\}$, and
symbols used as tags should always be set in roman type.  The following
sample theorem illustrates how to code for roman type within the
statement of a theorem.

\begin{theorem}
Let $\cal G$ be a free nilpotent-of-class-$2$ group of rank
$\ge 2$ with carrier $G$ and let
$$m : G\times G \to Z$$
satisfy \rom{(2.21)}, \rom{(2.22)}, and \rom{(2.24)}, and define
$\kappa$ by \rom{(2.23)}.  Then this kappa-group is kappa-nilpotent
of class $2$ and kappa-metabelian, that is to say, it satisfies
\rom{S2} and \rom{S3}, but it is kappa-abelian if, and only if,
\begin{equation}
m(x,y) = -1\quad\text{for all $x, y \notin G'$}.
\end{equation}
\rom{(}Thus \rom{(3.1)} implies the trivial consequence
\rom{(2.1)}.\rom{)}  Assume now that \rom{(3.1)} does not hold,
so that the kappa-group is kappa-nonabelian.  Assume further that $m$
is not constant outside $G'$ \rom{(}inside $G'$ the values of $m$
clearly do not matter\rom{)}.  Then $\kappa$ is neither left nor right
linear, that is to say, neither \rom{S4} nor \rom{S5} holds:
\rom{I1} again holds, but none of \rom{I2--I5}.  As before,
\rom{I6} is equivalent to \rom{(2.25)}.  Now \rom{I7$'$}, however,
is equivalent to a condition similar to \rom{(2.25)}, namely
\begin{equation}
m(xz\sigma, yz\sigma) = m(x,y)\,.
\end{equation}
\end{theorem}

Letters used as abbreviations rather than as variables or constants
are set in roman type.  Use the control sequences \cite[p.~95]{spivak:jot}
for common mathematical functions and operators like $\log$ and $\lim$,
and use \verb+\cite+ when citing a reference.  The reference tag
will be {\bf bold} automatically, but you will need to set any
additional information in roman type as illustrated by the coding of
the previous sentence.

\section{References}

To produce a bibliography, use the environment named
\verb+thebibliography+. Input each reference as you would normally do in
\LaTeX{} \cite{lamport:latex}. Arrange the references in alphabetical
order by the last name of the first named author.  The references at the
end of this sample file have been chosen to illustrate the coding of the
most common types of references. Use the abbreviations of names of
journals as given in annual indexes of {\it Mathematical Reviews}.
The sample references have been labeled with numbers, using
\verb+\bibitem{...}+. To get letter labels use, for example,
\verb+\bibitem[C1]{...}+.

References are set with hanging indentation.  The widest label should be
entered as the argument of the {\tt thebibliography} environment, if you
are not using Bib\TeX{} (which automatically determines the widest
label). For example, a bibliography containing more than a hundred
references would require three-digit number labels:
\par\begin{centering}\smallskip
\begin{verbatim}
\begin{thebibliography}{000}
\end{verbatim}
\smallskip\end{centering}

\section{Figures}

Figures to be inserted later should be handled using \LaTeX's \verb+figure+
environment.  The amount of space left should equal the exact height of the
figure.  Extra space around the figure will be provided automatically.  The
positioning of figures may need to be changed to obtain the best possible page
layout.  Thus it is important to label your figures and use the labels in the
text when referring to figures.  The figure caption should be positioned below
the figure.

In most cases, figures will be rendered for consistency of style within a
book.  Please provide figure manuscript drawn in black ink with clean,
unbroken lines on nonabsorbent paper.

\begin{example}
For the link in Figure~\ref{firstfig}, the Massey product $\langle u_1,
u_2, u_3, u_4, u_5\rangle$ in $S^3-L$ is defined and consists of all
integer multiples of $\gamma_{1,5}$.  For the link in
Figure~\ref{firstfig}, the Massey product $\langle u_1, u_2, u_3, u_4,
u_5\rangle$ in $S^3-L$ contains the single element $\gamma_{1,5}$. 
Since the links in Figures~\ref{firstfig} and~\ref{otherfig} are
homotopic, the example indicates that Massey products in $S^3-L$ with
distinct $u_j$'s do not, in general, determine homotopy invariants of
the link.  For the link in Figure~\ref{firstfig} and the link in
Figure~\ref{otherfig}, the Massey product $\langle u_1, u_2, \dots,
u_5\rangle$ in $\{S^3-L_i\}_{i=1}^5$ contains the single element
$\gamma_{1,5}$.
\end{example}

%  art work measures 11.5pc for figure 1, 7pc for figure 2

\begin{figure}[t]
\vskip 11.5pc
\caption{Only the word {\it figure} is set cap/small cap.  Any other words are
regular text.\label{firstfig}}
\end{figure}

\begin{figure}[t]
\vskip 7pc
\caption{\label{otherfig}}
\end{figure}


\section{Other headings}

\subsection{A subsection}  We conclude by noting that another characterization
of $A$-compactness follows from Mandelker. We call a family $\cal S$ of closed
sets in $X\ A$-stable if every $f\in A(X)$ is bounded on some member of $\cal
S$. Then one can show that a space is $A$-compact if and only if  every
$A$-stable family of closed sets with the finite intersection property has
nonempty intersection.

\subsubsection{A second-level subheading}

This paragraph is included only to illustrate the appearance of a
sub-subsection.

%%%%%%%%%%%%%%%%%%%%%%%%%%%%%%%%%%%%%%%%%%%%%%%%%%%%%%%%%

\bibliographystyle{amsplain}

\begin{thebibliography}{10}

\bibitem{arnold:sing}
 V. L. Arnol$'$d, A. N. Varchenko, and S. M. Gusein-Zade,
 {\em Singularities of differentiable maps.} I,
  ``Nauka'', Moscow, 1982 (Russian);
English transl. Birkh\"auser, 1985.

\bibitem{arnold:sing2}
\bysame,
 {\em Singularities of differentiable maps.}~II,
  ``Nauka'', Moscow, 1984;
English transl., Birkh\"auser, 1988.


\bibitem{bass:jacobian}
H. Bass, E. H. Connell, and D. Wright,
{\em The Jacobian conjecture}, Bull. Amer. Math. Soc.
 {\bf 7} (1982), 287--330.

\bibitem{bass:flows}
H. Bass and G. H. Meisters,
{\em Polynomial flows in the plane}, Adv. in Math.
 {\bf 55} (1985), 173--203.

\bibitem{coomes:injectivity}
B. Coomes,
{\em Polynomial flows, symmetry groups, and conditions sufficient for 
        injectivity of maps},
 Ph.D. Thesis, Univ. Nebraska-Lincoln, 1988.


\bibitem{coomes:lorenz}
\bysame,
{\em The Lorenz system does not have a polynomial flow},
{J. Differential Equations} (to appear).

\bibitem{formanek:gener}
E. Formanek,
{\em Generating the ring of matrix invariants},
Lecture Notes in Math., vol. 1197,
Springer-Verlag, Berlin and New York,
1986, pp. 73--82.

\bibitem{meisters:poly}
\bysame,
{\em Polynomial flows on $\hbox{\bf R}^n$},
Proc. Semester on Dynamical Systems (Warsaw, Autumn 1986),
Springer-Verlag,
Berlin, Heidelberg, and New York
(to appear).

\bibitem{osher:shock}
S. Osher,
{\em Shock capturing algorithms for equations of mixed type},
 Numerical Methods for Partial Differential Equations
(S. I. Hariharan and T. H. Moulton, eds.),
  Longman, New York, 1986, pp. 305--322.

\bibitem{ostro:nonlin}
L. A. Ostrovsky,
{\em Nonlinear internal waves in a rotating ocean},
Part 2, Oceanology {\bf 18} (1978), 181--191.

\bibitem{petrov:ellip}
G. S. Petrov,
{\em Elliptic integrals and their nonoscillatory behavior},
Funktsional. Anal. i Pri\-lo\-zhen. {\bf 20} (1986), 46--49;
English transl. in Functional Anal. Appl. {\bf 20} (1986).

\bibitem{spivak:jot}
M. D. Spivak,
{\em The Joy of \TeX{}}, second edition, Amer. Math. Soc., Providence, R.~I., 
1990.

\bibitem{lamport:latex} Leslie Lamport, {\em \LaTeX{} -- A Document Preparation
System}, Addison-Wesley, Reading, Mass., 1986.

\end{thebibliography}

\end{document}

%------------------------------------------------------------------------------
% End of dimacs-l.tex
%------------------------------------------------------------------------------
%------------------------------------------------------------------------------
% Beginning of dimacs-l.checklist
%------------------------------------------------------------------------------

		     Check List for Electronic Manuscripts

				  DIMACS Volumes


- The author should proofread the paper before submitting to the AMS.

- The title on the first page should have the first letter of major words
  and any proper names, in uppercase (CAPS); other words should be lowercase.

- The running heads for the right-hand (odd-numbered pages) should have
  the title (shortened if necessary) all in CAPS.

- The author(s) name(s) on the first page should be in CAPS and lowercase.

- The running heads for the left-hand (even-numbered pages) should have the
  author(s) name(s) all in CAPS.  Names in running heads should match names
  as given on the first page (shorten to fit only if necessary).

- The subject classification numbers should be listed as the first unmarked
  footnote on the first page.

- Grant information and final publication information should be listed as
  unmarked footnotes on the first page.

- Citations in the manuscript should be coded using "\cite".

- References should include all available information.

- The address, current address (if different) and e-mail address of each author
  should be included.

- The .tex files should include NO "input" files.

- All definitions should be  at the top of the file and none of them should
  redefine any LaTeX commands.

- Definitions should be used consistently throughout the paper.

- No definitions should be used for formatting text; only the LaTeX 
  environments should be used.

- The file should contain no line and/or page breaks.

- A covering letter should be included with the electronic submission.  The 
  letter should clearly identify the author, title and intended publication
  series.  It should also tell us where the author can be reached throughout
  the publication process.  

%------------------------------------------------------------------------------
% End of dimacs-l.checklist
%------------------------------------------------------------------------------

